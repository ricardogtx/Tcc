\begin{resumo}[Abstract]
 \begin{otherlanguage*}{english}
Difficulties and limitations are encountered by individuals with motor, physical
disabilities and also to systemic diseases, articular, neurological and muscular,
 sequelae to trauma and accidents. Thus have repercussions on the structure and
 also on the individual's body function, globally, as well as to influence the
activity and  participation of the same in the society. To decrease or to solve such problems
 a motor rehabilitation is necessary. For an effective rehabilitation an effective
 assessment is needed to identify the deficits found in the individual's dysfunction
 and thus provide appropriate treatment. Therefore, the measure range of motion
 is a decisive criterion in the physical therapy monitoring.  The objective of this work is the proposal of a software
  product that helps physiotherapists to prescribe motor rehabilitation programs, and allows patients to execute such programs
Without physical supervision.
   \vspace{\onelineskip}

   \noindent
   \textbf{Key-words}: motor rehabilitation, remote rehabilitation, serious game.
 \end{otherlanguage*}
\end{resumo}
