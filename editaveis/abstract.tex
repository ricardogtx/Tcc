\begin{resumo}[Abstract]
 \begin{otherlanguage*}{english}
Difficulties and limitations are encountered by individuals with motor, physical
disabilities and also to systemic diseases, articular, neurological and muscular,
 sequelae to trauma and accidents. Thus have repercussions on the structure and
 also on the individual's body function, globally, as well as to influence the
activity and  participation of the same in the society. To decrease or to solve such problems
 a motor rehabilitation is necessary. For an effective rehabilitation an effective
 assessment is needed to identify the deficits found in the individual's dysfunction
 and thus provide appropriate treatment. Therefore, the measure range of motion
 is a decisive criterion in the physical therapy monitoring. There are some solutions
 to aid in measuring the range of motion and motor rehabilitation, each with its
 specificity and limitation. The objective of this work is development of a system in a more affordable
 product both in terms of cost and mobility and convenience. Thus it is going to be developed
 a product proposal, which reduces the complexity and cost of a measurement of the
 range of motion and that can provide support on all phases of the motor rehabilitation.
   \vspace{\onelineskip}

   \noindent
   \textbf{Key-words}: motor rehabilitation. assessment. measuring the range of motion.
 \end{otherlanguage*}
\end{resumo}
