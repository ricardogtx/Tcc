\chapter[Considerações finais]{Considerações finais}\label{ch:conclusao}
  Como foi observado ao longo deste trabalho, podemos lidar com problemas de saúde e sociais com o auxílio da tecnologia,
e para tal não é necessário despendiar quantias massivas de dinheiro. A filosofia \textit{open-source} ajuda nesse quesito, incetivam a cooperação das pessoas
sem a necessidade de pagamentos, com isso podemos atacar tais problemas. Infelizmente por questões  técnicas não foi realizável desenvolver este trabalho apenas com
tecnologias que compartilham essa mesma filosofia.

  Foi possível o aprendizado em diversas áreas da tecnologia, da engenharia de software e alguns conceitos sobre a fisioterapia,
embora esta, seja uma área muito extensa e necessita de um profissional capacitado onde dificilmente a tecnologia irá substituir.
Com o decorrer do desenvolvimento, também foram apresentadas inúmeras adversidades, o que foi apresentado claramente em \ref{ch:problemas},
porém medidas foram tomadas e tais adversidades acabaram por contornadas.

  Corroboramos também que o \textit{software}, não pertence a área da engenharia onde um projeto especificado  é garantia de sucesso e cumprimento de todas as especifições. Ao longo do desenvolvimento
os requisitos sempre estão mudando, seja para adequação de alguma conformidade com a funcionalidade ou seja devido alguma barreira da tecnologia utilizada.

  Ao final, vimos que o sistema gerado com este trabalho é um produto acessível para a reabilitação motora, porém em casos onde não é ncessáro
fazer movimentos muito complexos e nem que exiga auxílio de objetos. É recomendável no uso da reabilitação de crianças onde o procedimento
fica menos tedioso e mais lúdico.
