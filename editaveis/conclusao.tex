\chapter[Conclusão]{Conclusão}\label{ch:conclusao}
  Como foi observado ao longo deste trabalho, podemos lidar com problemas de saúde e sociais com o auxílio da tecnologia,
e para tal não é necessário despendiar quantias massivas de dinheiro. A filosofia \textit{open-source} ajuda nesse quesito, incetivam a cooperação das pessoas
sem a necessidade de pagamentos, com isso podemos atacar tais problemas. Infelizmente por questões  técnicas não foi possível desenvolver este trabalho apenas com
tecnologias que compartilham essa filosofia.

  Ao final, foi visto que o sistema gerado com este trabalho tem indícios de futuramente poder virar um produto acessível para a reabilitação motora, porém em casos onde não é necessário
fazer movimentos muito complexos e nem que exijam auxílio de objetos. É recomendável no uso da reabilitação de idosos onde o procedimento
fica menos tedioso e mais lúdico.

Esta pesquisa obteve importantes resultados, em relação a implementação da solução, identificando
formas de auxílio a um problema de saúde e a problemas sociais, além de identificar uma nova maneira de utilizar o produto da microsoft, o
\textit{kinect}, e a viabilidade da aplicação no
contexto fisioterapêutico.

Para que a pesquisa se complete, deve-se implementar o algoritmo do Roberto de Souza Baptista \cite{roberto} como forma
de análise de movimento, onde não foi possível durante o desenvolvimento. Desse modo, deve-se destacar as lacunas
deixadas pela pesquisa, apresentando os trabalhos futuros para que novos interessados
possam seguir e completar o estudo e o desenvolvimento do sistema.
\begin{itemize}
  \item Implementar a arquitetura completa \ref{sol:arquiCompleta}, junto ao algoritmo do Baptista \cite{roberto} como módulo de processamento;
  \item Aplicar contextos de design no sistema, para mais engajamento ao uso;
  \item Possível aplicação de gameficação;
  \item Disponibilizar em alguma plataforma de distribuição de software.
\end{itemize}
