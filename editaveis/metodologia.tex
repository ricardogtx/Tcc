
\chapter[Metodologia]{Metodologia}
  Este capítulo irá descrever a metodologia utilizada no desenvolvimento deste trabalho.
Desse modo este capítulo está dividido nas seções Caracterização da pesquisa(\ref{sec:caracPesquisa}),
Metodologia Adotada no Desenvolvimento do sistema(\ref{sec:metDesenv}) e Planejamento das atividades(\ref{Sec:Planejamento}).

\section{Caracterização da Pesquisa}\label{sec:caracPesquisa}

Toda e qualquer classificação se faz mediante algum critério. Com relação às
pesquisas, é usual a classificação com base em seus objetivos gerais. Assim, é
possível classificar as pesquisas em três grandes grupos: exploratórias, descritivas e
explicativas\cite{gil}.

A classificação deste trabalho em relação aos objetivos da pesquisa é definida como
exploratória. A pesquisa exploratória tem como objetivo principal o aprimoramento de
ideias ou a descoberta de intuições, já a descritiva a descrição das características de
determinada população ou fenômeno e as explicativas identificar os fatores que
determinam ou que contribuem para a ocorrência dos fenômenos\cite{gil}.

O uso da pesquisa exploratória, neste trabalho, se dá pela necessidade do conhecimento
 sobre a avaliação de movimento, com o objetivo de
analisar a viabilidade do sistema no auxílio da reabilitação motora.


\section{Metodologia Adotada no Desenvolvimento do sistema}\label{sec:metDesenv}
  Para o desenvolvimento do sistema, adotamos algumas práticas ágeis e o \textit{kanban}\cite{kanban}
 para gerenciamento. O \textit{kanban} e o \textit{Product Backlog}, fonte das histórias de usuário,
   pode ser encontrado no \textit{trello}(\ref{sub:trello}).

  As \textit{Sprints} tiveram duração de duas semanas e foram
discutidas com os responsáveis por desempenhar os papéis de \textit{Product Owner}.


\subsection{Planejamento das atividades}
\label{Sec:Planejamento}
  Esta seção, descreve um cronograma de todas atividades
. Cronograma  é uma representação de um plano de
execução das atividades do trabalho, incluindo  outras
informações de planejamento.

O TCC, segundo as regras da UnB, é realizado em duas fases: uma chamada de TCC\_01 e outra chamada de TCC\_02.
 Durante o TCC\_01, o foco desse trabalho foi estabelecer os pilares teóricos para embasamento do projeto como um todo.
 Já durante o TCC\_02, o trabalho estará focado desenvolver um sistema para análise motora, atingindo uma prova de conceito,
  visando a viabilidade da proposta, com base nas fundamentações teóricas - conquistados ao longo do TCC\_01.

O TCC\_01 pôde ser sub-dividido em sete atividades: \textit{selecionar tema}, \textit{realizar pesquisa bibliográfica},
 \textit{definir proposta}, \textit{escrever referencial teórico}, \textit{estabelecer suporte tecnológico},
  \textit{evoluir metodologia} e \textit{apresentar TCC 1}.
  As mesmas estão distribuídas de acordo com o cronograma disposto na Tabela \ref{tab:cronograma1}.

\begin{table}[H]
	\centering
	\caption{Cronograma de atividades TCC\_1.}
	\label{tab:cronograma1}
	\begin{tabular}{@{}ccccc@{}}
		\toprule
		\textbf{Cronograma}             & \textbf{Março} & \textbf{Abril} & \textbf{Maio} & \textbf{Junho} \\ \midrule
		Selecionar Tema                 & X              &                &               &                \\ \midrule
		Realizar pesquisa bibliográfica & X              & X              & X             & X              \\ \midrule
		Definir proposta                & X              &                &               &                \\ \midrule
		Escrever referencial teórico    &                & X              & X             &                \\ \midrule
		Estabelecer suporte tecnológico &                & X              & X             & X              \\ \midrule
		Evoluir metodologia             &                & X              & X             &                \\ \midrule
		Apresentar TCC 1                &                &                &               & X              \\ \bottomrule
	\end{tabular}
\end{table}

Já a segunda etapa do trabalho, durante a realização do TCC\_2, o cronograma de atividades segue o apresentado na Tabela \ref{tab:cronograma2}.

\begin{table}[H]
	\centering
	\caption{Cronograma de atividades TCC\_2.}
	\label{tab:cronograma2}
	\begin{tabular}{@{}cccccc@{}}
		\hline
		\textbf{Cronograma}                                                           & \textbf{Março} & \textbf{Abril} & \textbf{Maio} & \textbf{Junho} & \textbf{Julho} \\ \hline
		\begin{tabular}[c]{@{}c@{}}Configurar\\Kinect/PC e ambiente\end{tabular}      & X               & X                          &                  &                   &                   \\ \hline
		Desenvolver solução \ref{historias}                                            &                 & X                          & X                & X                 &                   \\ \hline
		Testar solução                                                                &                 & X                          & X                & X                 &                   \\ \hline
		Analisar resultados                                                           &                 &                            &                  & X                 & X                 \\ \hline
		Apresentar TCC\_2                                                           &                 &                            &                  &                  & X                 \\ \hline
	\end{tabular}
\end{table}

\begin{itemize}
	\item \textbf{Propor tema}:

		A atividade de propor tema engloba desde a escolha do contexto em que se deseja trabalhar, até a definição dos orientadores do trabalho. Após a escolha do contexto e dos orientadores, buscou-se definir um escopo que será abordado durante o trabalho, ou seja, o tema. Os orientadores devem validar o tema escolhido para concluir a atividade.

	\item \textbf{Levantar bibliografia base}:

		Esta atividade refere-se à definição de pilares para o estudo proposto, ou seja, estabelecer o marco teórico do trabalho. Este levantamento garante o entendimento do contexto trabalhado e as possibilidades de atuação, especificando mais adequadamente o escopo.

	\item \textbf{Definir proposta}:

		Documentar a proposta de pesquisa para este trabalho. A proposta inclui, não apenas, mas, principalmente, uma introdução com a contextualização do tema, o objetivo geral e específicos, a justificativa e uma metodologia de pesquisa.

	\item \textbf{Realizar pesquisa e análise bibliográfica}:

		A pesquisa bibliográfica foi feita a partir do exame de qualificação do Roberto \cite{roberto}

	\item \textbf{Realizar referencial teórico}:

		O mesmo descreve o referencial teórico do trabalho em andamento.

	\item \textbf{Estabelecer suporte tecnológico}:

		Nesta atividade, são definidas as principais ferramentas e tecnologias utilizadas para a execução deste trabalho.

	\item \textbf{Apresentar TCC\_1}:

		Apresentar os resultados obtidos até o momento para a banca examinadora.

  \item \textbf{Desenvolver solução}:

    Desenvolvimento das histórias de usuário\ref{historias} levantandas ao longo do TCC\_1.

  \item \textbf{Testar solução}:

    Esta atividade ocorre em sicronia com a atividade de desenvolvimento, é essencial para analisar acordo com as histórias de usuário.

  \item \textbf{Analisar resultados}:

    Registrar e analisar informações relevantes que foram obtidas ao longo do desenvolvimento.

  \item \textbf{Apresentar TCC\_2}:

    Apresentar os resultados obtidos para a banca examinadora.
\end{itemize}
