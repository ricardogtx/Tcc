\chapter[Problemas enfrentados]{Problemas enfrentados}\label{ch:problemas}
  Este capítulo tem como apresentar os problemas enfrentados durante o desenvolvimento,
  assim como das soluções tomadas. Desse modo, este
capítulo está dividido nas seções: Multiplataforma (\ref{pro:Multiplataforma}),
Arquitetura(\ref{pro:arquitetura}) e Conhecimento das tecnologias utilizadas(\ref{pro:conhecimento}).
\section{Multiplataforma}\label{pro:Multiplataforma}
  Em proposta apresentada na entrega do primeiro documento, foi apresentado que a solução seria multiplataforma,
apresentando todas as bibliotecas possíveis para tal feito. Porém ao tentar instalar bibliotecas \textit{open soureces}
para a comunicação do PC com o sistema operacional Linux, distribuição Debian, notou-se que as bibliotecas não
estavam mais sendo mantidas, com atualizações datando de mais de 3 anos.

  Contudo ainda fez-se várias tentativas e por muito tempo,
a instalação e comunicação com o kinect e Linux, todavia havia dependências de pacotes em versões que não eram mais
mantidas no repositório do Debian, assim buscou-se alternativas para instalação de tais pacotes, aumentando o efeito cascata
de dependências de pacotes não mais mantidos, despendindo muito tempo e como resultado insucesso.

  Por final decidiu-se usar somente o sistema operacional da Microsoft o windows, com uma facilitada comunicação
kinect e sistema operacional através de \textit{SDK(Software Development Kit)}


\section{Arquitetura}\label{pro:arquitetura}
  Em proposta apresentada na entrega do primeiro documento, foi mostrada uma arquitetura de módulos, onde o core,
o processamento seria feito em python e comunicaria com os demais cores que poderiam ser acoplados. Uma arquitetura
inicial foi desenvolvida em python e pode ser encontrada \href{https://gitlab.com/ricardogtx/tcc2}{aqui}. Tal arquitetura conseguia
se comunicar com o Unity 3D, porém isso só acrescentaria mais complexidade tanto ao desenvolvimento, quanto a
instalação em ambiente de clientes.
  Por final decidiu-se usar somente o Unity3D com a linguagem C\#. Assim podemos contar com o poder e todos os recursos do C\#
 e das bibliotecas do Unity 3D que facilitam alguns cálculos como por exemplo,
 cálculos entre os ângulos das juntas. E se houver a necessidade, podemos utilizar os recursos
\textit{multiprocessing} do C\#.

\section{Conhecimento das tecnologias utilizadas}\label{pro:conhecimento}
  O autor pesquisador e desenvolvedor do sistema, não detinha conhecimento prévio de nenhuma das
tecnologias utilizadas, o Unity 3D e na linguagem C\#, com isso a velocidade do desenvolvimento ficou afetada
e possivelmente alguma decisão arquitetural não tenha sido a melhor.
