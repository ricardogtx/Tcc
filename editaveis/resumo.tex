\begin{resumo}
 %O resumo deve ressaltar o objetivo, o método, os resultados e as conclusões
 %do documento. A ordem e a extensão
 %destes itens dependem do tipo de resumo (informativo ou indicativo) e do
 %tratamento que cada item recebe no documento original. O resumo deve ser
 %precedido da referência do documento, com exceção do resumo inserido no
 %próprio documento. (\ldots) As palavras-chave devem figurar logo abaixo do
 %resumo, antecedidas da expressão Palavras-chave:, separadas entre si por
 %ponto e finalizadas também por ponto. O texto pode conter no mínimo 150 e
 %no máximo 500 palavras, é aconselhável que sejam utilizadas 200 palavras.
 %E não se separa o texto do resumo em parágrafos.

Dificuldades e limitações são encontradas por indivíduos que apresentam deficiências
 motoras, físicas e também a doenças sistêmicas, articulares, neurológicas e musculares,
 sequelas a traumas e acidentes. Desse modo, repercutem na estrutura e função do corpo
 do indivíduo, de forma global, além de influenciar na atividade e participação do
 mesmo frente a sociedade. Para diminuir ou até sanar tais problemas é necessário uma
 reabilitação motora. Para uma reabilitação eficiente necessita-se de uma avaliação
 eficaz para identificar os déficits encontrado nas disfunções do indivíduo e assim
proporcionar um tratamento apropriado. Por conseguinte, a medida da amplitude de
 movimento é um critério determinante utilizado na avaliação e no acompanhamento
 fisioterapêutico. O objetivo deste trabalho é a proposta de um produto de software
 que auxilie fisioterapeutas a preescrever programas de reabilitação motora,
  e permita o paciente executar tais programas sem supervisão física.
 \vspace{\onelineskip}

 \noindent
 \textbf{Palavras-chaves}: reabilitação motora, reabilitação remota, jogo sério.
\end{resumo}
