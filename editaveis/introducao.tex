\chapter[Introdução]{Introdução}
\section{Contextualização}
\label{Sec:contextualizacao}
  Segundo a WCPT - \textit{World Confederation for Physical Therapy }  \cite{wcpt}, a Fisioterapia é
 a área da saúde responsável por desenvolver, manter e reabilitar as
  capacidades de mobilidade e funcionalidade das pessoas ao longo de toda a sua vida.

  Seu objetivo é identificar e maximizar a qualidade de vida
 e de potencial de movimento de cada pessoa, dentro das áreas da promoção, prevenção,
  tratamento/intervenção, habilitação e reabilitação da saúde.

  Tratando-se de reabilitação motora, lida-se com dificuldades e limitações encontradas por
  indivíduos que apresentam deficiências motoras e ou físicas, bem como,
  a doenças sistêmicas, articulares, neurológicas e musculares, sequelas a
  traumas e acidentes.

  Problemas mais graves, repercutem
  na estrutura e função do corpo do indivíduo, de forma global, além de
  influenciar na atividade e participação do mesmo frente a sociedade. Afeta
  também, o ambiente em que se encontra e a forma como ele lida com sua rotina.
  Dessa maneira, torna-se ainda mais importante a eficácia do tratamento
  fisioterapêutico.



\section{Definição do Problema}
\label{Sec:DefinicaoProblema}

  Para uma reabilitação eficiente, necessita-se de uma avaliação eficaz,
 para identificar os deficits encontrados nas disfunções do indivíduo e proporcionar
  um tratamento apropriado, e também a cadência na realização dos exercícios propostos pelo
  profissional.

  Contudo, este último é um desafio, principalmente
  pela ausência dos pacientes nas sessões de fisioterapia. Tal ausência se dá por
  diversos fatores, desde fatores socioeconômicos, já que sessões de fisioterapia
  não costumam ser acessíveis para todos níveis sociais, a questões de disponibilidade de tempo.

  Desse modo, vê-se a necessidade de uma soulção que
   auxilie no tratamento fisioterapêutico e facilite na cadência
  da realização dos movimentos do tratamento apropriado, não necessitando de
  supervisão profissional e que possa ser executado a qualquer momento.


\section{Soluções Existentes}
\label{Sec:SolucoesExistentes}
Entre algumas soluções para o auxílio da reabilitação motora estão:
\begin{itemize}

\item Kinect Training: É um jogo eletrônico para o console de videogame xbox
360, com o auxílio do kinect para captação do movimento, que tem como objetivo
auxiliar em determinados exercícios físicos, substituindo o \textit{personal trainer}.
  \begin{itemize}
  \item Dificuldade: A limitação desse programa que queremos superar é a
  análise da qualidade da execução do movimento. Assim podemos dar um feedback
  para o usuário seja na prática de exercícios de  condicionamento físico ou em
  sessões de fisioterapia.
  \end{itemize}

\item Goniometria: O termo é formado por duas palavras gregas, \textit{gonia},
que significa ângulo, e \textit{metron}, que significa medida. Referindo-se à
medida de ângulos articulares presentes nas articulações dos seres humanos.
O instrumento mais utilizado para medir a amplitude de movimento é o goniômetro
universal. Este pode ser de plástico ou metal e de diferentes tamanhos, mas com
 o mesmo padrão básico\cite{manualGoniometria}.
  \begin{itemize}
  \item Dificuldade: Os dados medidos variam entre os examinadores devido ao modo como
  é feito a avaliação, dando pouca confiabilidade para este sistema.
  \end{itemize}

\item Qualisys Track Manager (QTM): É um software para coleta de dados de um
sistema que captura o movimento. O software integra com placas de força, EMG
e uma série de outros dispositivos e permite medir o movimento, no ar
ou debaixo d'água, passivo ou ativo\cite{qtm}.
  \begin{itemize}
  \item Dificuldade: Sistema pouco acessível para pequenas e médias clínicas,
  sendo necessário também um espaço grande e muitos sensores de custo elevado.
  \end{itemize}

\item Inclinômetro Digital: O inclinômetro digital é um instrumento da
engenharia para medir inclinação (em graus) de superfícies, após ser captada
por sensores sensíveis a gravidade\cite{erroMedicao2012}
  \begin{itemize}
  \item Dificuldade: Assim como o goniômetro os dados medidos variam entre os
  examinadores, porém com uma disparidade de valores um pouco menor.
  \end{itemize}

\end{itemize}

\section{Solução Proposta}
\label{Sec:SolucaoProposta}
  A partir do contexto apresentado, observando a dificuldade da regularidade dos
   exercícios, e o alto custo de soluções mais precisas, o
objetivo deste trabalho é tomar como inspiração a tese de doutorado do Roberto de Souza
Baptista, em um produto mais acessível
tanto em termos de custo, quanto a mobilidade e praticidade.

  O trabalho apresentado pelo Baptista \cite{roberto}, investiga técnicas para análise automática do
movimento humano. Assim, apresenta uma nova contribuição de um novo
procedimento para avaliação automática do movimento humano, que executa
segmentação e extração de parâmetros de desempenho motor, em séries temporais
de medidas de uma sequência de movimentos.

  O sistema proposto, apresenta três etapas, sendo elas:
  \begin{enumerate}
  \item Etapa de aquisição dos dados: movimento do usuário ser capturado pelo sistema (leitura dos ângulos das articulações):
    \begin{enumerate}
    \item Realizar o movimento de referência;
    \item Salvar arquivo referente ao movimento realizado.
    \end{enumerate}
  \item Etapa de importação do arquivo de movimento:
  \begin{enumerate}
      \item Importar o movimento.
  \end{enumerate}
  \item Etapa de visualizar e praticar os movimentos:
    \begin{enumerate}
    \item Visualizar lista com os movimentos cadastrados;
    \item Visualizar e praticar o movimento, de acordo com a animação do avatar;
    \item Usuário realiza o movimento selecionado;
    \item Sistema lê os movimentos das articulações do usuário em tempo real;
    \item Comparar as articulações de referência com os realizados pelo usuário;
    \item Mostra no avatar 3D as discrepâncias entre o movimento desejado e o movimento realizado.
    \end{enumerate}
  \end{enumerate}
\section{Objetivos}
\label{Sec:Objetivos}
\subsection{Objetivos Gerais}
\label{Sub:ObjetivosGerais}
  Desenvolver uma proposta de um produto de software,
  que auxilie fisioterapeutas a preescreverem programas de reabilitação motora,
   e permita o paciente executar tais programas sem supervisão física.
\subsection{Objetivos específicos}
\label{Sub:ObjetivosEspecificos}
\begin{itemize}

\item Levantar os requisitos da solução;
\item Propor uma solução de software;
\item Apresentar a viabilidade da proposta.

\end{itemize}

\section{Organização do Trabalho}

Este trabalho de conclusão de cusro está organizado em capítulos, cada capítulo contendo
seções, sendo eles:
\begin{itemize}
  \item Introdução: Capítulo referente à contextualização, justificativa e definição dos objetivos do trabalho;

			\item Revisão teórica: O objetivo deste capítulo é fornecer ao leitor o conhecimento necessário para compreender o trabalho realizado.

			\item Suporte tecnológico: Apresenta as ferramentas e tecnologias utilizadas para auxiliar o desenvolvimento deste trabalho,
       desde a pesquisa bibliográfica e documentação, até o desenvolvimento do sistema e apresentação;

			\item Metodologia: Este capítulo busca apresentar as técnicas utilizadas para a realização do trabalho,
       definindo as atividades a serem desempenhadas para conclusão do trabalho;

      \item O sistema: Neste capítulo é apresentado o produto desenvolvido,
       assim como as características e configuração do mesmo.

			\item Análise dos Resultados: Neste capítulo são apresentados os testes para análise
da viabilidade da utilização.

      \item Conclusão: Capítulo responsável pelas conclusões do desenvolvimento do trabalho;

			\item Considerações finais: Capítulo com o objetivo de apresentar as considerações finais do trabalho.
    \end{itemize}
