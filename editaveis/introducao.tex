\chapter*[Introdução]{Introdução}
\addcontentsline{toc}{chapter}{Introdução}

\section{Definição do Problema}
\label{Sec:DefinicaoProblema}

  Na reabilitação motora lida-se com dificuldades e limitações encontradas por
indivíduos que apresentam deficiências motoras e ou físicas , bem como,
 a doenças sistêmicas, articulares, neurológicas e musculares, sequelas a
traumas e acidentes.  Desse modo, essas alterações repercutem
 na estrutura e função do corpo do indivíduo, de forma global, além de
influenciar na atividade e participação do mesmo frente a sociedade. Afeta
também o ambiente em que se encontra e a forma como ele lida com sua rotina.
Dessa maneira, torna-se ainda mais importante a eficácia do tratamento
fisioterapêutico.

  Vale ressaltar que  a reabilitação eficiente depende de uma avaliação eficaz
 para identificar os déficits encontrados nas disfunções do indivíduo e assim
proporcionar um tratamento apropriado. Com isso, a medida da amplitude de
movimento é  um critério determinante utilizado na avaliação e no
acompanhamento fisioterapêutico. Para tal trabalho encontra-se disponíveis
diversos instrumentos e equipamentos para avaliação do movimento que
não são realísticos no cenário da prática clínica, e são geralmente usados em
pesquisas, grandes centros de reabilitação e universidades.
  Desse modo, quando refere-se a atenção de média complexidade, a avaliação do movimento por
esses profissionais torna-se subjetivo, descritivo e pouco
quantitativo.

\section{Soluções Existentes}
\label{Sec:SolucoesExistentes}
Entre algumas soluções para o auxílio na mensuração da amplitude do movimento
e na reabilitação motora estão:
\begin{itemize}

\item Kinect Training: É um jogo eletrônico para o console de videogame xbox
360, com o auxílio do kinect para captação do movimento, que tem como objetivo
auxiliar em determinados exercícios físicos, substituindo o personal trainer.
  \begin{itemize}
  \item Dificuldade: A limitação desse programa que queremos superar é a
  análise da qualidade da execução do movimento. Assim podemos dar um feedback
  para o usuário seja na prática de exercícios de  condicionamento físico ou em
  sessões de fisioterapia.
  \end{itemize}

\item Goniômetria: O termo é formado por duas palavras gregas, \textit{gonia},
que significa ângulo, e \textit{metron}, que significa medida. Referindo-se à
medida de ângulos articulares presentes nas articulações dos seres humanos.
O instrumento mais utilizado para medir a amplitude de movimento é o goniômetro
universal. Este pode ser de plástico ou metal e de diferentes tamanhos, mas com
 o mesmo padrão básico\cite{manualGoniometria}.
  \begin{itemize}
  \item Dificuldade: Os dados medidos variam entre os examinadores devido ao modo como
  é feito a avaliação,  dando pouca confiabilidade para este sistema.
  \end{itemize}

\item Qualisys Track Manager (QTM): É um software para coleta de dados de um
sistema que captura o movimento. O software integra com placas de força, EMG
e uma série de outros dispositivos e permite medir o movimento, no ar
ou debaixo d'água, passivo ou ativo\cite{qtm}.
  \begin{itemize}
  \item Dificuldade: Sistema pouco acessível para pequenas e médias clínicas,
  sendo necessário também um espaço grande e muitos sensores de custo elevado.
  \end{itemize}

\item Inclinômetro Digital: O inclinômetro digital é um instrumento da
engenharia para medir inclinação (em graus) de superfícies, após ser captada
por sensores sensíveis a gravidade\cite{erroMedicao2012}
  \begin{itemize}
  \item Dificuldade: Assim como o goniômetro os dados medidos variam entre os
  examinadores, porém com uma disparidade de valores um pouco menor.
  \end{itemize}

\end{itemize}

\section{Solução Proposta}
\label{Sec:SolucaoProposta}
  A partir do contexto apresentado, observando a dificuldade da mensuração da
amplitude do movimento (ADM), e o alto custo de soluções mais precisas, o
objetivo deste trabalho é transformar a tese de doutorado do Roberto de Souza
Baptista, também coorientador deste trabalho, em um produto mais acessível
tanto em termos de custo, quanto a mobilidade e praticidade.

  O trabalho apresentado investiga técnicas para análise automática do
movimento humano. Assim, apresenta uma nova contribuição de um novo
procedimento para avaliação automática do movimento humano que executa
segmentação e extração de parâmetros de desempenho motor em séries temporais
de medidas de uma sequência de movimentos.

  O sistema funcionará em 4 etapas, sendo elas:
  \begin{enumerate}
  \item Etapa de aquisição dos dados: movimento do usuário ser capturado pelo sistema (leitura dos ângulos das articulações)
  \item Etapa de treinamento:
    \begin{enumerate}
    \item Realizar o movimento de referência.
    \item Rotular os movimentos desejados.
    \item Salvar os rótulos de interesse.
    \item Realizar o treinamento dos modelos estatísticos e salvar as informações.
    \end{enumerate}
  \item Etapa de visualizar movimentos cadastrados
    \begin{enumerate}
    \item Visualizar lista com os movimentos cadastrados e rotulados.
    \item Visualizar animação de um movimento rotulado selecionado.
    \item Selecionar movimento(s) para prática.
  \end{enumerate}
  \item Etapa Prática:
    \begin{enumerate}
    \item Usuário realiza os movimentos selecionados na etapa anterior
    \item Sistema lê os movimentos das articulações do usuário em tempo real
    \item Comparar as articulações de referência com os realizados pelo usuário
    \item Mostra no avatar 3d as discrepâncias entre o movimento desejado e o movimento realizado
    \item Opção de salvar o movimento realizado
    \end{enumerate}
  \end{enumerate}
\section{Objetivos}
\label{Sec:Objetivos}
\subsection{Objetivos Gerais}
\label{Sub:ObjetivosGerais}
  Desenvolver um produto que diminua a complexidade e o custo de uma mensuração
precisa da amplitude do movimento e que possa dar auxílio e um feedback mais
confiável em todas as fases do processo da reabilitação motora.

\subsection{Objetivos específicos}
\label{Sub:ObjetivosEspecificos}
\begin{itemize}

\item Realizar estudo bibliográfico sobre as tecnologias que desejam-se
utilizar no trabalho, sendo elas:
  \begin{itemize}
    \item Unity 3d: Para a visualização da movimentação e feedback para o usuário.
    \item C\#: Para toda parte de tratamento e processamento dos dados e por ser uma linguagem suportada pelo Unity 3d.
  \end{itemize}

\item Efetuar estudo prévio no trabalho do Roberto de Souza Baptista.

\item Desenvolver o sistema.

\end{itemize}

%\section{Organização do Trabalho}
